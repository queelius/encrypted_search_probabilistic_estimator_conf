\documentclass[ ../main.tex]{subfiles}
\providecommand{\mainx}{..}
\begin{document}
\section{Conclusions and future work}
\label{sec:conclusions}
The primary contributions in this paper are two-folds. First, there has
not been much work for studying how safe encrypted searches are against
frequency attacks, which can be measured by a large number of attackers
for long period of time, possibly infinitely long. We provide studies on
the resilience of encrypted searches against frequency attacks from the
view point of resilience engineering approach to enhance security on
encrypted searches. Resilience engineering is a new way of enhancing
safety by precisely estimating the level of possible threats to a system
and feeding them back to adjusting or re-designing the system to
maintain the acceptable level of safety\cite{ref17}.

Our second contribution is development of a new method, Moving Average
Bootstrap (MAB) method, which efficiently and accurately calculates the
estimator for the minimum number of encrypted words ($N^*$) an
adversary needs to achieve a given accuracy level ($p^*$) with a
certain level of confidence as soon as a relatively small number of
samples ($n$) (i.e., encrypted words) are submitted by legitimate
users. Thus, the MAB method will let the defenders calculate the
estimator at an early stage without waiting for a large number of
queries submitted by legitimate users. Especially from the view point of
``tractability'', calculating the estimator using, not to mention an
infinitely large number of encrypted words, a large number of encrypted
words takes time (waiting for a large number of encrypted words to be
submitted) and huge storage (storage space to hold the submitted
encrypted words) is required.

Our proposed MAB method calculated the estimated number of encrypted
search queries an adversary needs to observe ($N^*$) for achieving a
given accuracy level, $p^* = 0.30$, at the confidence level of $95\%$
using only $5\%$ of the actual observations (250/5000) (Figure 5 (c)).
Assuming that the increase in the time an adversary needs to achieve a
certain $p^*$ is proportional to the ratio in the increase of the
number of the encrypted words observed by an adversary ($n$) for a
large number of encrypted words, the MAB method would allow a defender
to estimate $N^*$ in $5\%$ of time (without waiting for legitimate
users to issue a large number of encrypted words). We are currently
performing analyses using higher $p^*$ ($0.55$ through $0.80$) for
different levels of confidence ($90$ to $98\%$) for observing how they
affect the performance of MAB method and for observing if there is any
pathological case for MAB method.
\end{document}